\documentclass[a4paper, 12pt]{article}
\usepackage[utf8]{inputenc}
\renewcommand\familydefault{\sfdefault}
\usepackage[T1]{fontenc}
\usepackage[francais]{babel}
\usepackage[left=2cm,top=2cm,right=2cm,bottom=2cm]{geometry}
\usepackage{graphicx}
\usepackage{minted}
\usemintedstyle{colorful}
\usepackage{float}
\floatplacement{figure}{H}
\usepackage{authblk}
\usepackage{enumitem}
\setlist[enumerate]{label*=\arabic*.}
\usepackage{hyperref}
\hypersetup{
    colorlinks,
    citecolor=black,
    filecolor=black,
    linkcolor=black,
    urlcolor=blue
}

\usepackage{caption}
\newenvironment{code}{\captionsetup{type=listing}}{}

\begin{document}

\title{YourQuiz - Projet Logiciel}
\author{Raed Abdennadher, Mayron Bouchet, Matthieu Constant, \\ Nicolas Denby, Nathanaël Nufer, Steven Liatti}
\affil{\small Projet Logiciel - Profs. Yassine Rekik, Stéphane Malandain, Nabil Abdennadher}
\affil{\small Hepia ITI 3\up{ème} année}
\maketitle

\begin{figure}
	\begin{center}
		\includegraphics[width=0.85\textwidth]{mockups/png/mainPage.png}
	\end{center}
\end{figure}
\newpage

\tableofcontents
\listoffigures
% \renewcommand\listoflistingscaption{Table des scripts de code}
% \listoflistings

\newpage

\section{Analyse}
\subsection{Définition des rôles}
\subsubsection{Utilisateur anonyme}
\begin{itemize}
    \item Inscription
    \item Connexion
    \item Visualiser les quiz disponibles
\end{itemize}
\subsubsection{Utilisateur connecté}
\begin{itemize}
    \item Créer un quiz
    \item S'inscrire à un quiz
    \item Participer à un quiz
    \item Supprimer un quiz
    \item Visualiser le classement
\end{itemize}
\subsection{Définition des besoins}
\subsubsection{Exigences du projet}
\begin{itemize}
    \item Les rôles des utilisateurs
    \item Réalisation d'un client Web et Mobile
    \item Déploiement Cloud automatique
    \item Mettre en place des tests de performance
\end{itemize}
\subsubsection{Exigences fonctionnelles}
Les fonctions et les services offerts par le logiciel :
\begin{itemize}
    \item Création de compte
    \item Login
    \item Création d'un quiz
    \item Participer à un quiz
    \item Voir la liste des quiz
    \item Voir le classement
\end{itemize}

\newpage
\section{Règles de fonctionnement du site}
\subsection{Participation}
La structure des questions est la suivante : pour une question, il y aura plusieurs réponses possibles et une seule bonne réponse. Une limite maximale de temps par question est définie à 10 secondes.
\bigbreak
L'utilisateur est obligé de répondre à la question pour pouvoir continuer. Si l'utilisateur n'a pas répondu à la question avant la fin du temps imparti la réponse est considérée comme fausse et il passe automatiquement à la question suivante. Lorsqu'il répond à la question, on lui indique s'il a correctement répondu et on passe à la question suivante. L'utilisateur ne peut pas revenir en arrière. Il peut voir son score en tout temps et son avancement dans le questionnaire.
\bigbreak
Un système de coefficient multiplicateur par bonne réponse est présent. Au fur et à mesure que l'utilisateur répond juste aux questions, un coeffcient augmente jusqu'à une réponse fausse qui remet ce coefficient bonus à 0.
\bigbreak
Un questionnaire ne peut pas être refait. Un système de ranking des quiz et joueurs de quiz est disponible.
\subsection{Création}
N'importe quel utilisateur authentifié peut créer un quiz, mais il ne peut pas participer à ses propres quiz. L'utilisateur choisi le nom, la description, le nombre de questions et les réponses possibles.

\newpage
\section{Modélisation}
\begin{figure}
	\begin{center}
		\includegraphics[width=0.5\textwidth]{diagrams/infrastructure.png}
	\end{center}
    \caption{Infrastructure}
\end{figure}
\subsection{Diagrammes d'activités}
\subsubsection{Login}
\begin{figure}
	\begin{center}
		\includegraphics[width=0.5\textwidth]{diagrams/login.png}
	\end{center}
    \caption{Login}
\end{figure}
\subsection{Diagramme de cas d'utilisation}
\begin{figure}
	\begin{center}
		\includegraphics[width=1.0\textwidth]{diagrams/UseCaseQuiz.png}
	\end{center}
    \caption{Use case}
\end{figure}

\subsection{Maquettes}
\begin{figure}
	\begin{center}
		\includegraphics[width=0.73\textwidth]{mockups/png/mainPage.png}
        \caption{mainPage}
	\end{center}
\end{figure}
\begin{figure}
	\begin{center}
		\includegraphics[width=0.73\textwidth]{mockups/png/notMemberHome.png}
        \caption{notMemberHome}
	\end{center}
\end{figure}
\begin{figure}
	\begin{center}
		\includegraphics[width=0.73\textwidth]{mockups/png/enregistrement.png}
        \caption{enregistrement}
	\end{center}
\end{figure}
\begin{figure}
	\begin{center}
		\includegraphics[width=0.73\textwidth]{mockups/png/logIn.png}
        \caption{logIn}
	\end{center}
\end{figure}
\begin{figure}
	\begin{center}
		\includegraphics[width=0.73\textwidth]{mockups/png/inQuestion.png}
        \caption{inQuestion}
	\end{center}
\end{figure}
\begin{figure}
	\begin{center}
		\includegraphics[width=0.73\textwidth]{mockups/png/afterQuestion.png}
        \caption{afterQuestion}
	\end{center}
\end{figure}
\begin{figure}
	\begin{center}
		\includegraphics[width=0.73\textwidth]{mockups/png/myQuiz.png}
        \caption{myQuiz}
	\end{center}
\end{figure}
\begin{figure}
	\begin{center}
		\includegraphics[width=0.73\textwidth]{mockups/png/myParticipations.png}
        \caption{myParticipations}
	\end{center}
\end{figure}
\begin{figure}
	\begin{center}
		\includegraphics[width=0.73\textwidth]{mockups/png/createQuiz.png}
        \caption{createQuiz}
	\end{center}
\end{figure}
\begin{figure}
	\begin{center}
		\includegraphics[width=0.73\textwidth]{mockups/png/createQuestion.png}
        \caption{createQuestion}
	\end{center}
\end{figure}
\begin{figure}
	\begin{center}
		\includegraphics[width=0.73\textwidth]{mockups/png/mainPage.png}
        \caption{mainPage}
	\end{center}
\end{figure}

\section{Choix technologiques}
Le serveur sera fait avec \href{https://nodejs.org/en/}{Node.js} et \href{http://expressjs.com/}{ExpressJS} pour ses routes.
Il fournira des endpoints HTTP REST. Nous pensions utiliser \href{http://www.passportjs.org/}{PassportJS}, de concert avec 
ExpressJS pour l'authentification des utilisateurs.
La base de données reposera sur \href{https://www.mongodb.com/}{MongoDB}, 
pour la facilité d'intégration avec Node.js, la simplicité d'utilisation et la scalabilité. 
Les clients seront faits en HTML + Javascript pour le navigateur et en Java pour Android.

\section{User stories}
\begin{enumerate}
    \item Utilisateur anonyme
    \begin{enumerate}
        \item En tant qu'utilisateur anonyme, je veux pouvoir consulter les informations des quiz actuels.
        \item En tant qu'utilisateur anonyme, je veux pouvoir consulter le classement général de YourQuiz.
        \item En tant qu'utilisateur anonyme, je veux pouvoir m'inscrire au site YourQuiz.
    \end{enumerate}
    \item Utilisateur authentifié
    \begin{enumerate}
        \item En tant qu'utilisateur non-authentifié, je veux pouvoir m'authentifier sur le site YourQuiz, afin d'avoir accès à son contenu.
        \item En tant qu'utilisateur authentifié, je veux pouvoir afficher/modifier mon profil personnel.
        \item En tant qu'utilisateur authentifié, je veux pouvoir participer à un quiz d'un autre membre.
        \item En tant qu'utilisateur authentifié, je veux pouvoir consulter mes résultats par rapport à un quiz auquel j'ai participé.
        \item En tant qu'utilisateur authentifié, je veux pouvoir consulter ma place dans le classement générale, ainsi que les X meilleurs participants.
        \item En tant qu'utilisateur authentifié, je veux pouvoir créer un quiz.
        \item En tant qu'utilisateur authentifié, je veux pouvoir visualiser le nombre de quiz que j'ai créé.
        \item En tant qu'utilisateur authentifié, je veux pouvoir publier, modifier ou supprimer mes quiz.
        \item En tant qu'utilisateur authentifié, je veux pouvoir me déconnecter de mon compte YourQuiz.
    \end{enumerate}
\end{enumerate}

\newpage
\section{Planification du sprint 1}
Nous partons du principe qu'un sprint s'étale sur 1 semaine, avec 5 jours de 8 heures, ce qui représente 
40 heures. Nous sommes 5 à coder, cela fait 200 heures de temps total au maximum, nous pensons plutôt atteindre les 100 heures, 
avec les autres cours en parallèle et les phases de \textit{pair programming} et concertation. Nous pensions commencer par les 
User stories numéros 2.7 et 2.3 (Utilisateur authentifié). Ce sont les deux cas d'utilisation les plus importants.
\bigbreak
\textbf{2.7 - En tant qu'utilisateur authentifié, je veux pouvoir créer un quiz.}
\bigbreak
\begin{tabular}{|p{.6\linewidth}|c|c|} \hline
	\textbf{Tâche} & \textbf{Priorité} & \textbf{Temps (h)} \\ \hline
	Formulaire de création HTML + Javascript & 3 & 6 \\ \hline
	Endpoint serveur REST création de quiz & 3 & 8 \\ \hline
	Gestion de la base de données & 3 & 6 \\ \hline
	Design de la page HTML & 1 & 2 \\ \hline
	Activité application Android & 2 & 6 \\ \hline
    Authentification & 2 & 6 \\ \hline
    Tests & 2 & 8 \\ \hline
    Documentation & 1 & 4 \\ \hline
\end{tabular}

\bigbreak
\bigbreak
\textbf{2.3 - En tant qu'utilisateur authentifié, je veux pouvoir participer à un quiz d'un autre membre (lorsque la période pour effectuer le quiz est valide).}
\bigbreak
\begin{tabular}{|p{.6\linewidth}|c|c|} \hline
	\textbf{Tâche} & \textbf{Priorité} & \textbf{Temps (h)} \\ \hline
	Pages HTML + Javascript client & 3 & 10 \\ \hline
	Endpoint serveur REST check réponses & 3 & 6 \\ \hline
	Mise à jour base de données & 2 & 4 \\ \hline
	Activité application Android & 2 & 6 \\ \hline
    Tests & 2 & 8 \\ \hline
    Documentation & 1 & 4 \\ \hline
\end{tabular}

% \begin{code}
%     \inputminted[breaklines,breaksymbol=,linenos,frame=single,stepnumber=1,tabsize=2]{language}{code}
%     \caption{Légende}
%     \label{my_ref}
% \end{code}

% \href{https://www.vagrantup.com/}{Vagrant}

% \mintinline{text}{root}


\end{document}
